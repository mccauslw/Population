\documentclass[11pt,letter]{amsart}

\usepackage[margin=1.25in]{geometry}
\usepackage{graphicx}
\usepackage{amssymb}
\usepackage{amsmath}
\usepackage{epstopdf}
\usepackage{setspace}
\usepackage{currfile}
\usepackage[abbr]{harvard}
\usepackage[pagewise]{lineno}

\DeclareMathOperator{\Be}{Be}

\usepackage{hyperref}
\hypersetup{
    colorlinks,
    citecolor=black,
    filecolor=black,
    linkcolor=black,
    urlcolor=black
}

\begin{document}

\setpagewiselinenumbers
\modulolinenumbers[5]
\linenumbers
\doublespace
\setstretch{1.5}

Would it work if I prepared a (long) spreadsheet like this:

\vspace{0.5cm}
\begin{tabular}{cc|ccccc}
	Subject & Domain & $P_1$ & $P_2$ & $P_3$ & $P_4$ & $P_5$ \\
	\hline
	1 & 6 & 1 & 3 & 2 \\
	1 & 1 & 2 & 5 \\
	\vdots & \vdots \\
	1 & 23 & 5 & 1 & 3 & 2 & 4 \\
	\hline
	2 & 8 & 4 & 3 \\
	\vdots & \vdots \\
\end{tabular}
\vspace{0.5cm}

The first two lines mean the following:
Subject 1's first choice is from domain 6 (Juice) and the choice set consists of objects 1, 3 and 2 (Mango, Apple, Orange), listed (on the screen) in that order.
Subject 1's second choice is from domain 1 (Male Stars) and the choice set consists of objects 2 and 5 (Kevin Spacey and Christian Bale) in that order.

\bibliographystyle{oupecon}
\bibliography{bibliography}

\end{document}