\documentclass[11pt,letter]{amsart}

\usepackage{geometry}
\usepackage{graphicx}
\usepackage{amssymb}
\usepackage{epstopdf}
\usepackage{setspace}
\usepackage[abbr]{harvard}
%doublespace

\bibliographystyle{plain}

\begin{document}

\title{An experiment for population choice probabilities}
\author{William J. McCausland}
\date{Current version: \today.}

\maketitle

This document outlines some ideas for an experiment to observe choice behaviour of populations across a wide variety of choice situations.
A primary purpose of the experiment would be to test whether population choice probabilities satisfy the hypothesis of random utility.
A secondary purpose is to test other discrete choice axioms involving $m$-ary (i.e. not just binary) choice probabilities.
Two obvious examples are regularity and \possessivecite{SattTver76} multiplicative inequality.
Regularity is a necessary condition for random utility, and documented violations of random utility tend also to be violations of regularity.
The multiplicative inequality is a necessary condition for independent random utility and also for Elimination by Aspects.

We are looking for some suitable choice domains.
This list gives some examples of what we mean by choice domains:

\begin{enumerate}
	\item Perception (i.e. the sub-field of Psychology with that name)
	\begin{enumerate}
		\item Judging areas of rectangles. Objects would be rectangles with various lengths and widths.
		These are sometimes used in experiments on context effects.
		Most context effects constitute violations of random utility.
	\end{enumerate}
	\item Aesthetic judgement (i.e. agent chooses favourite object in a choice set)
	\begin{enumerate}
		\item Unfamiliar art forms
		(e.g. Australian aboriginal art to Canadians, Canadian first nations art to Australians)
		\item Book descriptions (from Amazon or dust jacket)
		\item Film descriptions (from DVD cover, film catalogue)
		\item Exotic travel destinations
		\item Dinner with various movie stars
	\end{enumerate}
	\item Consumer choice
	\begin{enumerate}
		\item Objects with attributes (e.g. mobile phones)
		\item Transportation choice
	\end{enumerate}
	\item Political choice
	\begin{enumerate}
		\item Policy options regarding the environment and other issues
		\item Choices among political parties and/or candidates
	\end{enumerate}
\end{enumerate}

We would like to gather examples from choice experiments already in the literature.
We are open to ideas for new choice domains, but existing experiments have certain advantages.
First, we can use the results of the existing experiments to calibrate choice objects so that choices probabilities are not too close to zero or one.
Second, we can reuse the instructions to participants and other aspects of the existing experimental design.
Finally, we will obtain results for choice situations for which there is demonstrated interest.

Given our research question, it is very important to observe choice probabilities on choice sets of different sizes (binary, ternary and $m$-ary choice).
\citeasnoun{Falm78} and \citeasnoun{Fior04} each give a set of necessary and sufficient conditions for random utility and these are both a collection of linear inequalities involving choice probabilities across choice sets of different sizes.
The set of conditions in \citeasnoun{Fior04} is the subset of indispensable conditions from \citeasnoun{Falm78}.
See also \citeasnoun{McCaMarl13} for further discussion of these issues in the context of testing random utility.

Our focus here is testing random utility and other behavioural conditions for {\em population} rather than {\em individual} choice probabilities.
A population choice probability is the probability that an agent drawn from some population chooses an object $x$ from a choice set $A$.
The random nature of the choice comes from both variation across individuals and random variation of an agent's choice across trials.
The latter variation may exist even if we only observe one trial.

We plan experiments with choice sets from multiple domains.
There would be about 20 {\em master sets} of five objects each.
Each master set would be a collection of similar objects from some choice domain,
and different master sets would come from different domains.
Choice sets are subsets, with at least two elements, of some master set.
Each participant would see only one subset from each master set, but would see subsets from a variety of master sets.
Different participants would see different choice sets from the same master set.
For a master set of size five, there are 26 subsets of cardinality greater than or equal to two, and many (perhaps 40) participants would see (and choose from) each subset.

The point of showing only one subset of each master set to a given participant is to ensure that choice probabilities really are independent and stable across trials.
This is in contrast with experiments involving repeated choice from an individual, where learning or increasing boredom are potential problems.

The fact that we are combining choice tasks from very different domains introduces some constraints.
First, the questions cannot be overly complicated.
In an experiment focussed on a single choice domain, the experimental can require participants to read  background material that is relevant to many different questions in an experiment.
We do not have this luxury.
Every choice situation needs to be easily explained and understood.
Second, we have to settle for stated preference experiments.
In some experiments, agents choose between physical objects or lotteries with cash prizes, and agents get to keep the objects they choose or the winnings of the lotteries they choose.
Often, to keep total payments to subjects reasonable, only some of the chosen objects or winnings are given to the participants.
Typically, this is done randomly, to give incentives to the participants to choose carefully.
In our experiment, we are going to have to settle for hypothetical choices.
Third, we need to get ethics approval and we would like to avoid copyright issues.
It is hard to justify controversial subjects or presenting materials protected by copyright.

%Here are some suggested papers to get started.
%\citeasnoun{RiesBuseMell06} mentions lots of choice experiments in the anomalies literature.
%The paper is directed at economists, but refers to the literature in psychology, economics and marketing.
%Most relevant is Section 5 on violations of regularity, since regularity is a necessary condition for random utility.
%\citeasnoun{BuseBarkMehtChat07} is more specifically about context effects.

\bibliographystyle{oupecon}
\bibliography{bibliography}

\end{document}
