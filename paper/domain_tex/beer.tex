This domain is from an experiment reported in \citeasnoun{HubePaynPuto82} used to illustrate an asymmetric dominance effect.
The prices are multiplied by 5 and we added choice objects d and e to allow for two more asymmetric dominance effects and two similarity effects.
The first panel of Figure \ref{f:CE} show the choice objects in attribute space.
Table \ref{t:CE} shows the relations of dominance, similarity and betweenness among objects associated with context effects.
The most commonly used domains to illustrate the attraction effect involve Beer, Cars, Apartments, Computers, Restaurants and Televisions.

\begin{tcolorbox}
Below you will find three brands of beer.
You know only the price per sixpack and the average quality ratings made by respondents in a blind taste test.
Given that you had to choose one brand to buy on this information alone, which one would you choose?

\begin{tabular}{cc}
\hline
Price/sixpack & Average quality rating (100 = Best; 0 = Worst) \\ 
\hline
\$9.00 & 50 \\ 
\$13.00 & 70 \\ 
\$15.00 & 70 \\ 
\$14.00 & 75 \\ 
\$15.00 & 80 \\ \hline
\end{tabular}
\end{tcolorbox}
