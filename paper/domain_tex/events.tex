This domain involves comparisons of the probabilities of future events.
Logically, the probability of event e must be as least as great as the probability of a, which must in turn be as least as great as the probability of d; also, the probability of b must be at least as great as the probability of c.
There is potential for several asymmetric dominance effects in this domain, based on these dominance relations.
These would be unconventional effets, as most experimental designs in the literature intended to elicit asymmetric dominance effects feature numerical dominance relations.

\begin{tcolorbox}
Which one of the following events do you think is most likely to happen in the next twenty years?

\begin{itemize}
	\setlength\itemsep{-5pt}
	\item Scotland becomes an independent country.
	\item Either Catalonia or Quebec become independent countries.
	\item Catalonia becomes an independent country.
	\item Scotland and Quebec become independent countries.
	\item Either Scotland or Quebec become independent countries.
\end{itemize}
\end{tcolorbox}
